\documentclass[11pt]{article}
\usepackage[utf8]{inputenc}
\usepackage[czech]{babel}
\usepackage{graphicx}

\title{Krásy počítačové grafiky: Úpravy rastrového obrazu}
\author{Tomáš Maršálek}
\date{9.\,března 2012}

\begin{document}
\maketitle

\section{Zadání}
Vyzkoušejte si naprogramovat metody úpravy digitalizovaného obrazu z přednášky,
jako je ostření, reliéf, warping, morphing.... Za program umějící aspoň jednu
techniku získáte 5 bodů, za každou další naprogramovanou techniku získáte max.
3 body podle obtížnosti, dohromady nejvýše 17 bodů. Program musí být schopen
načíst ze souboru obraz v rastrovém formátu a zase jej uložit, zobrazit původní
a změněný obraz s možností návratu o 1 akci. Odevzdáváte jako obvykle zdrojový
text, EXE a dokumentaci.

\section{Implementované filtry}
\subsection{Detekce hran}
Detekce hran je horní propust pro obrazový signál. Vysokou frekvencí v
rastrovém obrazu se rozumí velký rozdíl v intenzitě barvy v jednotlivých
kanálech.

\subsubsection{Sobel, Prewitt a Roberts Cross}
Jedná se o metody, které pro každý obrazový bod aplikují numerickou aproximaci
první derivace v tomto bodě. Pro dva rozměry se vypočte velikost první derivace
jako velikost gradientu, kde jednotlivé parciální derivace mají své příslušné
konvoluční matice. Vyšší změna barvy se projeví vyšší hodnotou derivace, což
vidíme jako detekovanou hranu. 

Například Sobelův operátor používá konvoluční matice: \\
$$
D_x = 
\left[
\begin{array}{ccc}
-1 & 0 & 1 \\
-2 & 0 & 2 \\
-1 & 0 & 1 \\
\end{array}
\right]
,~~~~
D_y = 
\left[
\begin{array}{ccc}
-1 & -2 & -1 \\
0 & 0 & 0 \\
1 & 2 & 1 \\
\end{array}
\right]
$$

\subsubsection{Rozdíl Gaussovských rozostření}
Rozostření funguje jako dolní propust pro obrazový signál. Největší změny
oproti původnímu obrázku nastanou právě v místech, kde se nachází hrany. Ty se
relativně rozostří nejvíce. Proto když od původního obrázku odečteme jeho
rozostřenou verzi, největší změnu uvidíme právě v místech hran.

Rozdíl dvou různě silných rozostření je pouze zobecnění rozdílu rozostření s
původním obrázkem.

\subsubsection{Laplacian of Sobel}
Laplaceův operátor je numerickou aproximací druhé derivace. Aplikací na rastr
získáme velmi tenké hrany, mnohem tenčí, než získané z výše uvedených metod.
Nevýhodou je, že je velmi citlivý na jakýkoliv šum. Proto některé filtry před
použitím Laplaceova operátoru odstraní šum, například Gaussovským rozostřením
(Laplacian of Gaussian) nebo Mediánovým filtrem.

Zde je Laplaceův operátor aplikován po Sobelově operátoru, výsledkem jsou velmi
tenké hrany s minimálním okolním šumem.

\subsection{Doostření}
Doostření obrázku je výsledkem součtu původního obrázku s jeho Laplaciánem.
Standardně je implementována možnost měnit intenzitu doostření pomocí
koeficientu ostření. \\

$$
B = A + c \cdot \Delta A
$$
$A$ je původní obrázek, $\Delta$ je Laplaceův operátor, $B$ je doostřený
obrázek a $c$ je koeficient ostření.

\subsection{Gaussovské rozostření}
% TODO

\end{document}
