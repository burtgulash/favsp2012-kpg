\documentclass[11pt]{article}
\usepackage[utf8]{inputenc}
\usepackage[czech]{babel}

\title{Krásy počítačové grafiky: Animace fraktálu}
\author{Tomáš Maršálek}
\date{10.\,března 2012}

\begin{document}
\maketitle
\section{Zadání}
Vytvořte animací libovolného fraktálu získaného z libovolného, ale vámi
vytvořeného programu změnami koeficientů, ukládáním jednotlivých dílčích
obrázků ve formátu .BMP a jejich dodatečným spojením pomocí programu BMP2AVI.
Odevzdáváte výslednou animaci plus program, ze kterého vznikla, spolu s
informací o potřebném nastavení parametrů pro zopakování tvorby této animace.
Bitmapy můžete ukládat ručně pomocí clipboardu nebo prostudovat formát BMP a
ukládat příslušný formát do souboru přímo z programu, ale spíše bych vám
doporučila prozkoumat, jestli ve vašem překladači není nějaká hotová
komponenta, která to umí. Za tuto úlohu získáte 10 bodů.

\section{Provedení}
Video je animací přiblížení u okraje Mandelbrotovy množiny až na samou hranici
přesnosti floating point čísel s dvojitou přesností (double). Změna parametrů
je v tomto případě velikost přiblížení a navíc rotace obrazovky pro estetický
efekt.

\section{Implementace}
Aplikace, která generuje animaci, je krátký program v Javě s napevno
zakódovanými parametry. Její účel je pouze vygenerovat tohle video. Výstupem je
420 snímků v bezztrátovém formátu png, které jsou pak pomocí kompozičního
nástroje programu Blender spojeny do videa formátu 720p uloženého v kontejneru
avi.

Snímky se počítají paralelně ve čtyřech vláknech s 16x supersamlingem. Pomocí
rekurzivního algoritmu na rozdělení plátna a detekci černých míst jsou
eliminovány zbytečné výpočty pro místa, kde by docházelo k použití plného počtu
iterací a zdlouhavému výpočtu, kvůli čemuž vznikají právě ona černá místa. Pro
zvýšení estetické kvality bylo použito vyhlazování barevných pásů.

Jedním problémem, se kterým se lze při hlubokém zoomování do nitra fraktálů
setkat, je ztráta barev, a tedy kontrastu, při velkém přiblížení. Jedním
používaným řešením je zmenšování palety barev stejným tempem, jakým probíhá
přibližování. Druhým, zde použitým, řešením je rekurzivní rozdělení barevné
palety. Barvy se v tomto případě nijak nepohybují, zůstávají na místě, ale čas
od času si v animaci všimneme opakování té samé palety.

\section{Závěr}
Program nesplňuje výše uvedené zadání doslova, protože zmíněný program bmp2avi
není vůbec použit. Nicméně stejného výsledku je dosaženo jiným způsobem.
\end{document}
