\documentclass[11pt]{article}
\usepackage[utf8]{inputenc}
\usepackage[czech]{babel}

\title{Krásy počítačové grafiky: Animace fraktálu}
\author{Tomáš Maršálek}
\date{10.\,března 2012}

\begin{document}
\maketitle
\section{Zadání}
Vytvořte animací libovolného fraktálu získaného z libovolného, ale vámi
vytvořeného programu změnami koeficientů, ukládáním jednotlivých dílčích
obrázků ve formátu .BMP a jejich dodatečným spojením pomocí programu BMP2AVI.
Odevzdáváte výslednou animaci plus program, ze kterého vznikla, spolu s
informací o potřebném nastavení parametrů pro zopakování tvorby této animace.
Bitmapy můžete ukládat ručně pomocí clipboardu nebo prostudovat formát BMP a
ukládat příslušný formát do souboru přímo z programu, ale spíše bych vám
doporučila prozkoumat, jestli ve vašem překladači není nějaká hotová
komponenta, která to umí. Za tuto úlohu získáte 10 bodů.

\section{Provedení}
Video je animací přiblížení u okraje Mandelbrotovy množiny až na samou hranici
přesnosti floating point čísel s dvojitou přesností (double). Změna parametrů
je v tomto případě velikost přiblížení a navíc rotace obrazovky pro estetický
efekt.

\section{Implementace}
Aplikace, která generuje animaci je krátký program v Javě s natvrdo
zakódovanými parametry. Její účel je pouze vygenerovat tohle video. Výstupem je
300 obrázků v bezztrátovém formátu .png, které jsou pak pomocí kompozičního
nástroje programu Blender spojeny do videa uloženého v kontejneru .avi.

Program implementuje několik dodatečných úprav pro zvýšení kvality výsledku:
\begin{itemize}
\item 16x supersampling
\item Rekurzivní barevná paleta
\item Vyhlazování barevných pásů pomocí potenciální funkce
\end{itemize}

\section{Závěr}
Program nesplňuje výše uvedené zadání doslova, zmíněný program bmp2avi není vůbec použit. Nicméně stejného výsledku je dosaženo jiným způsobem.

Možná vylepšení by byla zefektivnění výpočtu iterací. Existují divide \& conquer algoritmy, které se snaží odhalit \uv{černá místa}, která jsou výpočetně nejsložitější. Hezkým efektem pro oživení přibližovacích animací tohoto typu je posunování palety barev. Barvy se pak na okrajích fraktálu chovají jako kapalina, která se snaží vyplnit \uv{suchá místa}.
\end{document}
