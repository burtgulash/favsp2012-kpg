\documentclass[11pt]{article}
\usepackage[utf8]{inputenc}
\usepackage[czech]{babel}

\title{Krásy počítačové grafiky: Generování polygonu}
\author{Tomáš Maršálek}
\date{9.\,března 2012}

\begin{document}
\maketitle

\section{Zadání}
Vymyslete co nejefektivnější a implementačně nejjeednodušší algoritmus pro
vygenerování náhodného  obecného polygonu (takového, který má konvexní i
nekonvexní úhly a jeho strany se neprotínají).

\section{Postup}
Algoritmus začíná s nedegenerovaným trojúhelníkem. V každé iteraci vybere
náhodně hranu a bod na ní. Posunutím tohoto bodu o náhodnou nenulovou
vzdálenost ve směru kolmice k hraně, na které leží, získáme dvě nové hrany. Je
samozřejmě třeba zajstit, aby nedošlo k protnutí dvou nově vytvořených hran s
již existujícími. 

Protože vycházíme z polygonu, který splňuje zadání, a v každé iteraci provedeme
kroky, které zadání nijak neporuší, můžeme kdykoliv algoritmus zastavit a
budeme mít polygon vyhovující zadání.

\section{Protnutí hran}
Při nově vytvořených hranách může dojít k průsečíkům, čímž by se porušila
podmínka, že se strany nesmí protínat. Algoritmus v každé iteraci testuje
průsečíky dvou nových hran se všemi stávajícími hranami. Ve skutečnosti stačí
testovat jen pro jednu z nových hran, protože pokud by došlo k průniku, musí se
protnout obě.

Detekci protnutí hran lze algoritmitky řešit jako průsečík dvou úseček, které
jsou určeny parametrickými rovnicemi.
$$P_a = P_1 + t_a (P_2 - P_1),\ t_a \in [0, 1]$$
$$P_b = P_3 + t_b (P_4 - P_3),\ t_b \in [0, 1]$$ \\

Průsečík získáme porovnáním bodů $P_a$ a $P_b$, výsledné rovnice, které leží
pod vektorovou reprezentací, budou
$$x_1 + t_a (x_2 - x_1) = x_3 + t_b (x_3 - x_1)$$
$$y_1 + t_b (y_2 - y_1) = y_3 + t_b (y_3 - y_1)$$ \\

Symetrická řešení pro parametry $t_a$ a $t_b$ jsou zlomky, které mají stejný
jmenovatel
$$t_a = \frac{(x_4 - x_3)(y_1 - y_3) - (y_4 - y_3)(x_1 - x_3)} 
             {(y_4 - y_3)(x_2 - x_1) - (x_4 - x_3)(y_2 - y_1)}$$
$$t_b = \frac{(x_2 - x_1)(y_1 - y_3) - (y_2 - y_1)(x_1 - x_3)}
             {(y_4 - y_3)(x_2 - x_1) - (x_4 - x_3)(y_2 - y_1)}$$ \\

Pro úsečky stačí zjistit, jestli leží oba parametry $t_a$ a $t_b$ v intervalu
$[0, 1]$, pak víme, že se úsečky protly. Pokud jsou čitatel i jmenovatel nula,
úsečky jsou koincidentní.

V implementaci je možné provádět všechny operace na přesné celočíselné
aritmetice, proto není třeba řešit numerické chyby.

\section{Algoritmus}
Následující pseudokód popisuje klíčové prvky algoritmu.

\begin{verbatim}
vytvoř negenerující trojúhelník
opakuj (n - 3) krát:
    náhodně vyber hranu
    vyber na ní bod
    opakuj:
        vyber náhodnou hodnotu
        posuň bod i s přilehlými stranami o hodnotu ve směru kolmice
    dokud se nově vytvořené hrany neprotínají s existujícími
vykresli hotový n-gon
\end{verbatim}

Dobu běhu nemůžeme přesně určit, protože vnitřní cyklus, který tvoří nové
hrany, závisí na náhodném výběru hodnoty posunu. Pokud očekáváme pouze několik
iterací tohoto cyklu, doba běhu je řádu $O(n^2)$, protože n-krát testujeme
průsečík s n existujícími hranami.

\end{document}
